\documentclass[main.tex]{subfiles}
%% Current Author:
\setcounter{chapter}{11}
\begin{document}
\chapter{Electric Fields}
\begin{content}
  \item concept of an electric field
  \item uniform electric fields
  \item capacitance
  \item electric potential
  \item electric field of a point charge
\end{content}
\subsection{Candidates should be able to:}
\spec{explain what is meant by an electric field and recall and use $E=\frac{F}{q}$ for electric field strength}
\spec{recall that applying a potential difference to two parallel plates stores charge on the plates and produces a uniform electric field in the central region between them}
\spec{derive and use the equations Fd = QV and $E=\frac{V}{d}$ for a charge moving through a potential difference in a uniform electric field}
\spec{recall that the charge stored on parallel plates is proportional to the potential difference between them}
\spec{recall and use $C = <frac{Q}{V}$ for capacitance}
\spec{recognise and use $W = \frac{1}{2}QV$ for the energy stored by a capacitor, derive the equation from the area under a graph of charge stored against potential difference, and derive and use related equations such as $W = \frac{1}{2}CV^2$}
\spec{analyse graphs of the variation with time of potential difference, charge and current for a capacitor discharging through a resistor}
\spec{define and use the time constant of a discharging capacitor}
\spec{analyse the discharge of a capacitor using equations of the form $x=x_0e^{\frac{-t}{RC}}$}
\spec{understand that the direction and electric field strength of an electric field may be represented by field lines (lines of force), and recall the patterns of field lines that represent uniform and radial electric fields}
\spec{understand electric potential and equipotentials}
\spec{understand the relationship between electric field and potential gradient, and recall and use $E = -\frac{dV}{dX}$}
\spec{recognise and use $ F = \frac{Q_1 Q_2}{4\pi\epsilon_0 r^2} $ for point charges}
\spec{derive and use $E = \frac{Q}{4\pi\epsilon_o r} $ for the electric field due to a point charge}
\spec{*use integration to derive $W = \frac{Q_1 Q_2}{4\pi\epsilon_0 r}$ from $F=\frac{Q_1 Q_2}{4\pi\epsilon_0 r^2}$ for point charges}
\spec{*recognise and use $W = \frac{Q_1 Q_2}{4\pi\epsilon_0 r }$ for the electrostatic potential energy for point charges.}
\end{document}
