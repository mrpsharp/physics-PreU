\documentclass[main.tex]{subfiles}
%% Current Author:

\begin{document}
\chapter{Equations}

\section{Equations you need to remember for pre-U}

velocity v= ∆x/∆t

acceleration a=∆v/∆t

resultant force F=ma

momentum p=mv

resultant force F=∆p/∆t

impulse Impulse = change in momentum

density Density = mass/volume

pressure Pressure = force/area

pressure P = ρgh for a liquid

weight W = mg

power P = Fv

GPE ∆E = mg∆h

\begin{quote}
g∆h = change in gravitational potential
\end{quote}

energy in a spring E = ½Fx

efficiency \% efficiency =
\(\frac{\text{useful\ energy\ }\left( \text{or\ power} \right)\text{out}}{\text{total\ energy\ }\left( \text{or\ power} \right)\text{\ in}}\)
x 100

current I = ∆Q/∆t

potential difference V = W/Q

resistance R = V/I

electrical power P = VI

electrical work done W = VIt

resistance R = ρl/A

resistors in series R­\textsubscript{T} = R\textsubscript{1} +
R\textsubscript{2} + \ldots{}

resistors in parallel
\(\frac{1}{R_{T}} = \frac{1}{R_{1}} + \frac{1}{R_{2}} + \ldots\)

frequency f = 1/T

wave speed v = fλ

Malus' law Intensity α cos\textsuperscript{2}θ

photoelectric equation \(hf = \phi + \frac{1}{2}mv_{\max}^{2}\)

angular velocity v = rω

period T = 2π/ω

circular motion F = mv\textsuperscript{2}/r

electric field strength E = F/Q

capacitance C = Q/V

field strength/potential E = - dV/dX

the ideal gas law pV = nRT

Boltzmann factor \(e^{- \frac{E}{\text{kT}}}\)

activity A = - dN/dt

luminous flux \(F = \frac{L}{4\pi d^{2}}\)

Hubble's law \(v \approx H_{0}d\)

Hubble time t = 1/H\textsubscript{0}

\section{Equations you need to derive and remember for pre-U}

energy in a spring E = ½ kx\textsuperscript{2}

kinetic energy E = ½ mv\textsuperscript{2}

emf E = I(R+r)

emf E = V + Ir

electrical power P = I\textsuperscript{2}R

critical angle sin c = 1/n

centripetal acceleration a = v\textsuperscript{2}/r

centripetal acceleration a = rω\textsuperscript{2}

uniform electric field Fd = QV

uniform electric field E = V/d

energy in a capacitor W = ½ CV\textsuperscript{2 }

energy in a capacitor W = ½ Q\textsuperscript{2}/C

electric field due to a point charge
\(E = \ \frac{Q}{{4\pi\varepsilon_{0}r}^{2}}\)

Kepler's third law r\textsuperscript{3} α T\textsuperscript{2}

gravitational field strength \(g = \frac{\text{Gm}}{r^{2}}\)

radius of curvature of particle in B-field
\(r = \frac{\text{mv}}{\text{BQ}}\)

Hall effect V = Bvd

kinetic theory of gases pV =
\textsuperscript{1}/\textsubscript{3}Nm\textless{}c\textsuperscript{2}\textgreater{}

activity of a source A = λN

activity at time t A = A­­\textsubscript{0}e\textsuperscript{-λt}

half-life \(t_{\frac{1}{2}} = \frac{ln2}{\lambda}\)

\section{Equations you need to remember for paper 3, part B, but not for any
other part of the examination.}

moment of inertia \(I = \Sigma mr^{2}\)

shm \(\frac{d^{2}x}{dt^{2}} = - \omega^{2}x\)

x = Acosωt

\section{Equations you need to derive for paper 3, part B, but not for any other
part of the examination.}

moment of inertia of a ring I = mr\textsuperscript{2}

moment of inertia of a disc I = mr\textsuperscript{2}/2

moment of inertia of a rod about one end I = ml\textsuperscript{2}/3

moment of inertia of a rod about its centre I = ml\textsuperscript{2}/12

velocity in shm v = -Aωsinωt

acceleration in shm a = -Aω\textsuperscript{2}cosωt

total energy in shm E =
\textsuperscript{1}/\textsubscript{2}mA\textsuperscript{2}ω\textsuperscript{2}

electric potential (from electric force)
\(W = \frac{Q_{1}Q_{2}}{4\pi\varepsilon_{o}r}\)

hydrogen atom energy levels \(E_{n} = \frac{- 13.6eV}{n^{2}}\)
\end{document}