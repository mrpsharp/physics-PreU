\documentclass[main.tex]{subfiles}
%% Current Author: BC
\setcounter{chapter}{6}
\begin{document}
\chapter{Superposition}
\begin{content}
\item phase difference
\item diffraction
\item interference
\item standing waves
\end{content}

\section*{Candidates should be able to:}
\spec{explain and use the concepts of coherence, path difference, superposition and phase}
\spec{understand the origin of phase difference and path difference, and calculate phase differences from path differences}
\spec{understand how the phase of a wave varies with time and position}

These terms are all used when considering more than one wave.

The phase of a wave is related to how far through an oscillation a wave is. This is expressed in radians or degrees, where one complete oscillation corresponds to $360^{\circ}$ or $2\pi$ radians.

The phase difference between two waves is more useful. This refers to the fraction of an oscillation by which one wave 'leads' or 'lags' behine another.

\spec{determine the resultant amplitude when two waves superpose, making use of phasor diagrams}
\spec{explain what is meant by a standing wave, how such a wave can be formed, and identify nodes and antinodes}

A standing wave arises from a combination of reflection and interference. Consider the set up below. 

Diagram

The vibration generator leads to a progressive wave travelling to the right. This waves reflects off the fixed end and so there are now two waves on the string, travelling in opposite directions. 

These two waves superpose and, in some cases (conditions discussed below), a standing wave can form on the string. If these conditions are met, there will be points where the two waves always meet in phase and interfere constructively. These are called \emph{antinodes}. The points where the two waves always meet out of phase and interfere destructively are called \emph{nodes}.

Diagram

Unlike a progressive wave, all points on a stationary wave do not have the same amplitude. The amplitude is at a minimum (often zero) at a node and at a maximum at an antinode.

All types of waves can produce standing waves. The most common examples are:

\spec{understand that a complex wave may be regarded as a superposition of sinusoidal waves of appropriate amplitudes, frequencies and phases}

When two waves superpose, they can 

\spec{recall that waves can be diffracted and that substantial diffraction occurs when the size of the gap or obstacle is comparable to the wavelength}

\spec{recall qualitatively the diffraction patterns for a slit, a circular hole and a straight edge}

\spec{recognise and use the equation $n\lambda = b sin\theta$ to locate the positions of destructive superposition for single slit diffraction, where $b$ is the width of the slit}

When an electromagnetic wave, such as light, travels through a single slit, it will diffract. Therefore light from the slit reaches many points on a screen placed at some distance from the slit.

Consider point P on the screen. Light will reach point P from all points within the slit. The distances travelled from various points within the slit to point P will be different, and so there will be a path and phase difference between all of the waves arriving at point P.

Therefore, as we move along the screen, there will be points of constructive interference and points of destructive interference. 

It can be shown that for a single slit of width $b$, the points of \textbf{destructive interference} will be at an angle $\theta$ given by the equation
$n\lambda = b sin\theta$
Here, $n$, refers to the order of the minimum being considered.

\spec{recognise and use the Rayleigh criterion $\theta \approx \lambda$ b for resolving power of a single aperture, where $b$ is the width of the aperture}

As light travels through an instrument, such as the eye, or a telescope, it passes through a gap or aperture and will diffract. The diffraction pattern will depend on the shape of the aperture and the width.

If


\spec{describe the superposition pattern for a diffraction grating and for a double slit and use the equation $d \sin\theta = n\lambda$ to calculate the angles of the principal maxima}
\spec{use the equation $\theta = \frac{ax}{D}$ for double-slit interference using light}




\end{document}
