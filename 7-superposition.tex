\documentclass[main.tex]{subfiles}
%% Current Author: BC
\setcounter{chapter}{6}
\begin{document}
\chapter{Superposition}
\begin{content}
\item phase difference
\item diffraction
\item interference
\item standing waves
\end{content}

\section*{Candidates should be able to:}
\spec{explain and use the concepts of coherence, path difference, superposition and phase}

When two or more waves meet at a point,

\spec{understand the origin of phase difference and path difference, and calculate phase differences from path differences}
\spec{understand how the phase of a wave varies with time and position}
\spec{determine the resultant amplitude when two waves superpose, making use of phasor diagrams}
\spec{explain what is meant by a standing wave, how such a wave can be formed, and identify nodes and antinodes}



\spec{understand that a complex wave may be regarded as a superposition of sinusoidal waves of appropriate amplitudes, frequencies and phases}
\spec{recall that waves can be diffracted and that substantial diffraction occurs when the size of the gap or obstacle is comparable to the wavelength}
\spec{recall qualitatively the diffraction patterns for a slit, a circular hole and a straight edge}
\spec{recognise and use the equation $n\lambda = b sin\theta$ to locate the positions of destructive superposition for single slit diffraction, where $b$ is the width of the slit}
\spec{recognise and use the Rayleigh criterion $\theta \approx \lambda$ b for resolving power of a single aperture, where $b$ is the width of the aperture}
\spec{describe the superposition pattern for a diffraction grating and for a double slit and use the equation $d \sin\theta = n\lambda$ to calculate the angles of the principal maxima}
\spec{use the equation $\theta = \frac{ax}{D}$ for double-slit interference using light}

\end{document}
