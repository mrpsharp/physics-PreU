\documentclass[main.tex]{subfiles}
%% Current Author: PS
\setcounter{chapter}{2}
\begin{document}
\chapter{Deformation of Solids}
\section*{Content}
\begin{itemize}
\item elastic and plastic behaviour
\item stress and strain
\end{itemize}

\section*{Candidates should be able to:}
\spec{distinguish between elastic and plastic deformation of a material}

Elastic deformation is defined as deformation where the sample returns to its original length when the load is removed. Plastic deformation involved a permanent change in length of the sample.

\spec{recall the terms brittle, ductile, hard, malleable, stiff, strong and tough, explain their meaning and give examples of materials exhibiting such behaviour}

\begin{description}
    \item[Brittle] Brittleness is an indicator of how soon after the yield point a material fractures. Failure will be through the propagation of cracks. A brittle material cannot absorb much energy before breaking. For example, glass and ceramics can be strong but brittle.
    \item[Ductile] Ductility is a measure of plastic behaviour under tension. It gives an indication of how easily a material can be drawn into wires i.e. can withstand large strains without breaking. Copper is highly ductile.
    \item[Hard] Hardness is a measure of a materials ability to resist impact or scratching. Diamond is an exceptionally hard material.
    \item[Malleable] Malleability is a measure of plastic behaviour under compression. It gives an indication of how easily a material can be worked. Metals are ductile and hence relatively easy to form into shapes for use in manufacture.
    \item[Stiff] Stiffness relates the resistance to change of shape a material possesses. A measure of stiffness is the Young Modulus. Glass fibres and steel are both stiff materials.
    \item[Strong] A strong material is able to withstand a large stress without failing.
    \item[Tough] Toughness is a measure of the ability of a material to resist failure through crack propagation. It is the opposite of brittleness. A tough material is able to absorb a lot of energy without breaking. Plastics/polymers are often tough.
\end{description}

\spec{explain the meaning of, use and calculate tensile/compressive stress, tensile/compressive strain, spring constant, strength, breaking stress, stiffness and Young modulus}

Firstly, compressive forces and deformations are those which reduce the length of the sample whereas tensile forces act to increase its length.

\begin{description}
    \item[Stress] Stress is defined as the force per unit of cross-sectional area applied to a material. \[ \sigma = \frac{F}{A} \] Stress is measured in pascals (Pa).
    \item[Strain] Strain is the fractional extension of a material. \[ \epsilon = \frac{x}{l} \]
    where $l$ is the orinal length.
    \item[Spring constant] The spring constant $k$ is the force per unit of extension of a material during its proportional phase of deformation. It is defined by Hooke's Law: \[ F = kx\] The spring constant is often used as a measure of stiffness of an object.
    \item[Strength] Strength is often measured as the maximum stress a material can withstand before permanent deformation. This is known as the yield stress.
    \item[Breaking stress] This is the stress at which the material fails.
    \item[Young modulus] This is a quantitative measure of the stiffness of a material, defined as stress per unit of strain in the proportional region the material's behaviour. \[ E = \frac{\sigma}{\epsilon} \]
\end{description}

\spec{draw force-extension, force-compression and tensile/compressive stress-strain graphs, and explain the meaning of the limit of proportionality, elastic limit, yield point, breaking force and breaking stress}

The gradient of a force-extension graph gives the spring constant.

\begin{figure}[ht]
    \begin{center}
        \begin{tikzpicture}
            \draw[->] (-0.5,0) -- (10,0) node[anchor=north] {$\epsilon$};
            \draw[->] (0,-0.5) -- (0,6) node[anchor=east] {$\sigma$};
            \draw (0,0) -- (1.5,3) .. controls (2,3.5) .. (2.5,3) .. controls (3,2.8) .. (4,4) .. controls (5.5,5.5) and (8,5.5) .. (10,4.5);
            \fill (1.5,3) circle (2pt) node[anchor=south east] {A};
            \fill (2,3.4) circle (2pt) node[anchor=south] {B};
            \fill (7,5.2) circle (2pt) node[anchor=south] {C};
            \fill (10,4.5) circle (2pt) node[anchor=south] {D};
            \node[draw, text width = 4cm] at (13,3) {
            \begin{description}
                \item[A] The limit of proportionality
                \item[B] The yield point
                \item[C] The ultimate stress (maximium stress)
                \item[D] The breaking point
            \end{description}
            };
        \end{tikzpicture}
    \end{center}
    \caption{Stress-strain curve for a ductile material}
    \label{stress-strain}
\end{figure}

Figure \ref{stress-strain} shows an example stress-strain curve. Note that the limit of proportionality is often a good approximation of the elastic limit of a metal. The ``breaking stress'' usually refers to the ultimate stress, i.e. the maximum stress the material can withstand, rather than the stress at the breaking point. 

\spec{state Hooke’s law and identify situations in which it is obeyed}

\[ F = kx \]

Hooke's law is obeyed by an ideal spring and by a sample of metal up to the limit of proportionality.

\spec{account for the stress-strain graphs of metals and polymers in terms of the microstructure of the material.}


\end{document}
