\documentclass[main.tex]{subfiles}
%% Current Author:
\setcounter{chapter}{5}
\begin{document}
\chapter{Waves}
\begin{content}
\item progressive waves
\item longitudinal and transverse waves
\item electromagnetic spectrum
\item polarisation
\item refraction
\end{content}

\section*{Candidates should be able to:}
\spec{understand and use the terms displacement, amplitude, intensity, frequency, period, speed and wavelength}

\spec{recall and apply $f = \frac{1}{T}$ to a variety of situations not limited to waves}
\spec{recall and use the wave equation $v=f\lambda$}
\spec{recall that a sound wave is a Longitudinal wave which can be described in terms of the displacement of molecules or changes in pressure}
\spec{recall that light waves are transverse electromagnetic waves, and that all electromagnetic waves travel at the same speed in a vacuum}
\spec{recall the major divisions of the electromagnetic spectrum in order of wavelength, and the range of wavelengths of the visible spectrum}
\spec{recall that the intensity of a wave is directly proportional to the square of its amplitude}
\spec{use graphs to represent transverse and longitudinal waves, including standing waves}
\spec{explain what is meant by a plane-polarised wave}
\spec{recall Malus’ law ($I \propto \cos^2\theta $) and use it to calculate the amplitude and intensity of transmission through a polarising filter}
\spec{recognise and use the expression for refractive index
\[ n = \frac{\sin{\theta_1}}{\sin{\theta_2}} = \frac{v_1}{v_2}\]}
\spec{derive and recall $sin{C} = \frac{1}{n}$ and use it to solve problems}
\spec{recall that optical fibres use total internal reflection to transmit signals}
\spec{recall that, in general, waves are partially transmitted and partially reflected at an interface between media.}
\end{document}
