\documentclass[main.tex]{subfiles}
\begin{document}
\chapter{Introduction}
\section{Revising for the Pre-U}
\newpage
\section{Structure of Assessment}
\begin{table}[h]
  \begin{tabular}{|lr|c|}
    \hline
    \textbf{Components} & & \textbf{Weighting} \\
    \hline
    \textbf{Paper 1 Multiple Choice} & \textbf{1 hour 30 minutes} & 20\% \\
    \multicolumn{2}{|p{10cm}|}{Candidates answer 40 multiple-choice questions based on Parts A and B of the syllabus content.

40 marks} & \\ \hline
\textbf{Paper 2 Written Paper} & \textbf{2 hours} & 30\% \\
\multicolumn{2}{|p{10cm}|}{Section 1: Candidates answer structured questions based on Part A of the syllabus
content.

Section 2: Candidates answer structured questions related to pre-released material.

100 marks} & \\ \hline
\textbf{Paper 3 Written Paper} & \textbf{3 hours} & 35\% \\
\multicolumn{2}{|p{10cm}|}{Section 1: Candidates answer structured questions requiring short answers or calculations and some longer answers. The questions are focused on Part B of the syllabus content, but may also draw on Part A.

Section 2: Candidates answer three questions from a choice of six. Three questions will have a strong mathematical focus and three questions will focus on philosophical issues and/or physics concepts. Learning outcomes marked with an asterisk (*) will only be assessed in this section.

140 marks} & \\ \hline
\textbf{Practical Investigation} & \textbf{20 hours} & 15\% \\
\multicolumn{2}{|p{10cm}|}{Candidates plan and carry out an investigation of a practical problem of their own choosing. Candidates are assessed on their ability to: plan; make detailed observations of measurements; use a range of measuring instruments; use appropriate physics principles; and produce a well-organised report.

30 marks} & \\ \hline
  \end{tabular}
\end{table}
\end{document}
